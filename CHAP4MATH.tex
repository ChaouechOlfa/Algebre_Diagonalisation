\documentclass[a4paper,12pt,french]{article}
\usepackage{mathpazo}
\usepackage[utf8]{inputenc}
\usepackage[top=0.5cm, bottom=3cm, left=2cm, right=2cm]{geometry}
\geometry{headsep=7ex}
\usepackage{setspace}
\doublespacing
\usepackage{fancyhdr}
\usepackage{ mathtools }
\usepackage{graphicx}
\usepackage{array,multirow,makecell}
\usepackage{array,multirow,makecell}
\usepackage{tikz}
\usepackage{diagbox}
\pagenumbering{arabic}
\usepackage{graphicx}
\usepackage{diagbox}
\usepackage{tikz}
\usepackage{multirow}
\usepackage{draftwatermark}
\usetikzlibrary{plotmarks}
\usetikzlibrary{calc}
\usepackage{hyperref}
\usetikzlibrary{patterns}
\usepackage{geometry}
\date{}
\pagestyle{fancy}
\renewcommand{\headrulewidth}{0pt}
\SetWatermarkColor{gray!15}
\SetWatermarkFontSize{3cm}
\SetWatermarkText{Dr.Chaouech Olfa}
\headheight=30pt

\begin{document}
	
	\title{\textbf{Chapitre IV: Diagonalisation des matrices carrées}}
	\maketitle
	\section{Valeurs propres-vecteurs propres:}
	\subsection{Définition:}
	Soit A une matrice carrée d'ordre n, V un vecteur non nul appartient à $\mathbb{R}^n$ et $\lambda$ un scalaire. On dit que $V$ est un vecteur propre de $A$ associé à la valeur propre $\lambda$ si et seulement si: $AV=\lambda V$.
	\subsection{Polynôme caractéristique:}
	Soit A une matrice carrée d'ordre n. $\lambda$ est une valeur propre de A et X un vecteur propre de A associé à $\lambda$ donc on a:\\
	$AX=\lambda X$
	$\Leftrightarrow $$AX-\lambda X=0$	$\Leftrightarrow $$(A-\lambda I)X=0$\\
	Le polynôme caractéristique de A est par définition $P_{A}(\lambda)=det(A-\lambda I)$\\
	\textbf{Théorème:}\\
	Les valeurs propres d'une matrice carrée A d'ordre n sont les racines de polynôme caractéristique \textcolor{red}{$P_{A}(\lambda)=0$}.\\
	\textbf{Exemple 1:}\\
	Soit la matrice $A_{1}$ suivante:
	\[\begin{pmatrix}
		2 & 1  \\
		1 & 2 \\
		
	\end{pmatrix}\] 
	Déterminer les valeurs propres de $A_{1}$.\\
	Réponse:
	\begin{itemize}
		\item Cherchons le polynôme caractéristique de $A_{1}$\\
		${P_{A_{1}}(\lambda)=det(A_{1}-\lambda I_{2})=\begin{vmatrix}
				
				2-\lambda&1\\
				1&2-\lambda
			\end{vmatrix}=(2-\lambda)^2-1}$ 
		\item Pour déterminer les valeurs propres de $A$ il suffit de résoudre:\\
		$P_{A_{1}}(\lambda)=0$ 
	\end{itemize}
	\textbf{Exemple 2:}\\
	Soit la matrice $A_{2}$ suivante:
	\[\begin{pmatrix}
		1 & 1  \\
		0 & 1 \\
		
	\end{pmatrix}\] 
	\url{https://youtu.be/vrru5an0DRM}\\
	\textbf{Remarques:}
	\begin{itemize}
		\item On dit que $\lambda$ est une valeur propre de multiplicité $\alpha$, si $\lambda$ est une racine d'ordre $\alpha$ de $P_{A}(\lambda)=0$.
		\item Si $A$ est une matrice carrée d'ordre $n$, $\lambda_{i}$ est une valeur propre de $A$ de multiplicité $\alpha_{i}$ alors $n=\sum_{i=1}^{p}\alpha_{i}$ (l'ordre de la matrice est égale à la somme de multiplicité de ces valeurs propres).
		\item Une valeur propre de multiplicité 1 est dite valeur propre simple.
		\item Une valeur propre de multiplicité $\alpha >1$ est dite valeur propre multiple.
		\begin{itemize}
			\item Si $\alpha=2$: Valeur propre double.
			\item Si $\alpha=3$: Valeur propre triple.
		\end{itemize}
	\end{itemize}
	
	\textbf{Propositions:}
	\begin{itemize}
		\item Si $\lambda$ est une valeur propre non nulle d'une matrice inversible $A$, alors $\frac{1}{\lambda}$ est une valeur propre de $A^{-1}$.
		\item Si $\lambda$ est une valeur propre non nulle d'une matrice $A$, alors  $\lambda ^{p}$ est une valeur propre de $A^p$.
		\item Les valeurs propres d'une matrice triangulaire sont les éléments de sa diagonale principale.
		\item La somme des valeurs propres de $A$ = trace($A$).
		\item Le produit des valeurs propres de $A$ = $det(A)$.
	\end{itemize}
	\textbf{Exemples:}\\
	Donner les valeurs propres des matrices suivantes:
	\[B=\begin{pmatrix}
		3 & 0 & 0  \\
		1 & 3 & 0 \\
		0 & 1 & 2
		
	\end{pmatrix}	~~~~~~;~~~~~ C=\begin{pmatrix}
		1 & 0 & 0  \\
		0 & 2 & 0 \\
		0 & 0 & 4
		
	\end{pmatrix}\] 
	
	
	\subsection{Recherche des vecteurs propres:}
	\subsubsection{Théorème:}
	A une valeur propre $\lambda$, on associe une infinité des vecteurs propres tous colinéaires entre eux.\\ $V_{1}$, $V_{2}$ sont colinéaires : $V_{1}=\alpha V_{2}$\\
	\textbf{Exemple:}
	\[A=\begin{pmatrix}
		1 & 0 & 0  \\
		1 & 2 & 0 \\
		1 & 0 & -1
		
	\end{pmatrix}\] 
	Déterminer les vecteurs propres de $A$.\\
	Réponse\\
	$A$ est une matrice triangulaire donc ses valeurs propres sont:\\
	$\lambda_{1}=1$; $\lambda_{2}=2$; $\lambda_{3}=-1$\\\\
	$V=\begin{pmatrix}
		x  \\
		y  \\
		z
		
	\end{pmatrix} \in \mathbb{R}^{3}$, est un vecteur propre de $A$ associé à $\lambda_{1}=1$, s'il vérifie l'égalité suivante:\\
	\[AV=\lambda_{1}V \rightarrow (A-I_{3})V=0\] 
	\[\begin{pmatrix}
		0 & 0 & 0  \\
		1 & 1 & 0 \\
		1 & 0 & -2
		
	\end{pmatrix}.\begin{pmatrix}
		x   \\
		y   \\
		z 
		
	\end{pmatrix}=\begin{pmatrix}
		0   \\
		0  \\
		0 
		
	\end{pmatrix}\] 
	\[det(A-I_{3})V=0\]
	\[\left \{
	\begin{array}{rcl}
		x+y&=&0 \\
		x-2z&=&0
	\end{array}
	\right. \Leftrightarrow \left \{
	\begin{array}{rcl}
		y&=&-x \\
		z&=&\frac{1}{2}x\\
		x& \in& \mathbb{R}
	\end{array}
	\right.\Leftrightarrow \left \{
	\begin{array}{rcl}
		x&=&x \\
		y&=&-x\\
		z&=&\frac{1}{2}x ~~~,~~~x \in \mathbb{R}
		
	\end{array}
	\right.\]
	\[V=\begin{pmatrix}
		x  \\
		-x  \\
		\frac{1}{2}
		
	\end{pmatrix}=x.\begin{pmatrix}
		1  \\
		-1  \\
		\frac{1}{2}
		
	\end{pmatrix}=x.V_{1}~~~,~~~ x \in \mathbb{R}\]
	L'ensemble des vecteurs propres de $A$ associé à $\lambda_{1}$, sont les vecteurs colinéaires à $V_{1}$, cet ensemble s'appelle sous-espace propre de $\lambda_{1}$, noté $E(\lambda_{1})$ et puisque tous les vecteurs de $E(\lambda_{1})$ sont colinéaires à $V_{1}$, on dit que $E(\lambda_{1})$ est engendré par $V_{1}$ et dimension de $E(\lambda_{1})=1$, on écrit $dim E(\lambda_{1})=1$.\\
	$V=\begin{pmatrix}
		x  \\
		y  \\
		z
		
	\end{pmatrix} \in \mathbb{R}^{3}$, est un vecteur propre de $A$ associé à $\lambda_{2}=2$, s'il vérifie l'égalité suivante:\\
	\[AV=\lambda_{2}V \rightarrow (A-2I_{3})V=0\]
	\[\begin{pmatrix}
		-1 & 0 & 0  \\
		1 & 0 & 0 \\
		1 & 0 & -3
		
	\end{pmatrix}.\begin{pmatrix}
		x   \\
		y   \\
		z 
		
	\end{pmatrix}=\begin{pmatrix}
		0   \\
		0  \\
		0 
		
	\end{pmatrix}\] 
	\[det(A-2I_{3})V=0\]
	\[\left \{
	\begin{array}{rcl}
		-x&=&0 \\
		x&=&0\\
		x-3z&=&0
	\end{array}
	\right. \Leftrightarrow \left \{
	\begin{array}{rcl}
		x&=&0 \\
		y& \in& \mathbb{R}\\
		z&=&0
	\end{array}
	\right.\Leftrightarrow \]
	\[V=\begin{pmatrix}
		0  \\
		y  \\
		0
		
	\end{pmatrix}=y.\begin{pmatrix}
		0  \\
		1  \\
		0
		
	\end{pmatrix}=y.V_{2}~~~,~~~ y \in \mathbb{R}\]
	Le sous-espace propre de $E(\lambda_{2})$ est engendré par $V_{2} $ et $dim E(\lambda_{2}=2)$\\
	$V=\begin{pmatrix}
		x  \\
		y  \\
		z
		
	\end{pmatrix} \in \mathbb{R}^{3}$, est un vecteur propre de $A$ associé à $\lambda_{3}=-1$, s'il vérifie l'égalité suivante:\\
	\[AV=\lambda_{3}V \rightarrow (A+I_{3})V=0\]
	\[\begin{pmatrix}
		2 & 0 & 0  \\
		1 & 3 & 0 \\
		1 & 0 & 0
		
	\end{pmatrix}.\begin{pmatrix}
		x   \\
		y   \\
		z 
		
	\end{pmatrix}=\begin{pmatrix}
		0   \\
		0  \\
		0 
		
	\end{pmatrix}\] 
	\[det(A+I_{3})V=0\]
	\[\left \{
	\begin{array}{rcl}
		2x&=&0 \\
		x+3y&=&0\\
		x&=&0
	\end{array}
	\right. \Leftrightarrow \left \{
	\begin{array}{rcl}
		x&=&0 \\
		y&=&0\\
		z&\in& \mathbb{R}
	\end{array}
	\right. \]
	\[V=\begin{pmatrix}
		0  \\
		0  \\
		z
		
	\end{pmatrix}=z.\begin{pmatrix}
		0  \\
		0  \\
		1
		
	\end{pmatrix}=z.V_{3}~~~,~~~ z \in \mathbb{R}\]
	Le sous-espace propre de $E(\lambda_{3})$ est engendré par $V_{3} $ et $dim E(\lambda_{3})=1$.\\
	\textbf{Exemple}\\
	Soit les matrices suivantes:
	\[A=\begin{pmatrix}
		2 & 1  \\
		1 & 2 
		
	\end{pmatrix}~~~; ~~~B=\begin{pmatrix}
		1 & 1  \\
		0 & 1 
		
	\end{pmatrix}~~~; ~~~C=\begin{pmatrix}
		15 & -2 & 6  \\
		21 & -2 & 9\\
		-28 & 4 & -11
		
	\end{pmatrix}
	\]
	Déterminer les vecteurs propres correspondants.
	\section{Diagonalisation des matrices:}
	\subsection{Définition:}
	$A$ est une matrice carrée, est diagonalisable s'il existe une matrice inversible \textcolor{red} {$P$}, dite \textcolor{red}{matrice de passage}, et une \textcolor{red}{ matrice diagonale} \textcolor{red}{D}. Tel que: \[A=P.D.P^{-1}\]
	\subsection{Théorème:}
	Pour qu'une matrice carrée $A$ soit diagonalisable, il faut et il suffit que la dimension de sous-espace propre associé à $\lambda_{i}$ égale à la multiplicité de cette valeur propre $\lambda_{i}$\\
	$A$ est \textcolor{red}{diagonalisable ssi:}
	\[dim E(\lambda_{i})=\alpha_{i}\] Avec $\alpha_{i}$: multiplicité de $ \lambda_{i}$\\
	\textbf{Remarque:}\\
	\begin{itemize}
		\item La matrice digonale $D$ est la matrice des valeurs propres.
		\item La matrice de passage $P$ est la matrice des vecteurs propes.
	\end{itemize}
	\subsection{Exemple:}
	\[A=\begin{pmatrix}
		2 & 1  \\
		1 & 2 
		
	\end{pmatrix}\]
	$\lambda_{1}=1$ : Valeur propre simple ($\alpha_{1}=1$).\\
	$\lambda_{2}=3$ : Valeur propre simple ($\alpha_{2}=1$).\\\\
	$V_{1}=\begin{pmatrix}
		1  \\
		-1  
		
	\end{pmatrix}$ ~~~;~~~ \textcolor{red}{$dim E(\lambda_{1})=1=\alpha_{1}$}.\\
	$V_{2}=\begin{pmatrix}
		1  \\
		1  
		
	\end{pmatrix}$ ~~~;~~~ \textcolor{red}{$dim E(\lambda_{2})=1=\alpha_{2}$}.\\
	\textbf{cl:} $A$ est diagonalisable, donc il existe $D$, matrice diagonale, et $P$,matrice de passage, tel que: 
	\[A=P.D.P^{-1}\]
	\[P=[V_{1} V_{2}]=\begin{pmatrix}
		1 & 1  \\
		-1 & 1 
		
	\end{pmatrix} ~~~;~~~ D=\begin{pmatrix}
		\lambda_{1} & 0  \\
		0 & \lambda_{2} 
		
	\end{pmatrix}=\begin{pmatrix}
		1 & 0  \\
		0 & 3 
		
	\end{pmatrix}\]
	\[P^{-1}=\frac{1}{2}.\begin{pmatrix}
		1 & -1  \\
		1 & 1
	\end{pmatrix} \]\\
	\[B=\begin{pmatrix}
		1 & 1  \\
		0 & 1 
		
	\end{pmatrix}\]
	$\lambda_{1}=1$ : Valeur propre double ($\alpha=2$).\\
	$V=\begin{pmatrix}
		1 \\
		0  
		
	\end{pmatrix}$;~~~ \textcolor{red}{$dim E(\lambda)=1\ne \alpha=2$}~~~$\Rightarrow B$ n'est pas diagonalisable.
	\[ C=\begin{pmatrix}
		1 & 0 & 0  \\
		1 & 2 & 0\\
		1 & 0 & -1
		
	\end{pmatrix}
	\]
	~~~ \\
	$\lambda_{1}=1$ : Valeur propre simple ($\alpha_{1}=1$).\\
	$\lambda_{2}=2$ : Valeur propre simple ($\alpha_{2}=1$).\\
	$\lambda_{3}=-1$ : Valeur propre simple ($\alpha_{3}=1$).\\~~~ \\
	$V_{1}=\begin{pmatrix}
		1  \\
		-1  \\
		\frac{1}{2}
		
	\end{pmatrix}$ ~~~;~~~ \textcolor{red}{$dim E(\lambda_{1})=1=\alpha_{1}$}.\\
	~~~ \\
	~~~ \\
	$V_{2}=\begin{pmatrix}
		0  \\
		1  \\
		0
		
	\end{pmatrix}$ ~~~;~~~ \textcolor{red}{$dim E(\lambda_{2})=1=\alpha_{2}$}.\\
	$V_{3}=\begin{pmatrix}
		0  \\
		0  \\
		1
		
	\end{pmatrix}$ ~~~;~~~ \textcolor{red}{$dim E(\lambda_{3})=1=\alpha_{3}$}.\\
	
	$C$ est diagonalisable donc il existe une matrice $D$ diagonale et une matrice $P$ tel que $C=P.D.P^{-1}$ avec:\\
	
	\[ D=\begin{pmatrix}
		1 & 0 & 0  \\
		0 & 2 & 0\\
		0 & 0 & -1
		
	\end{pmatrix} ~~~;~~~P=\begin{pmatrix}
		1 & 0 & 0  \\
		-1 & 1 & 0\\
		-\frac{1}{2} & 0 & 1
		
	\end{pmatrix}
	\]
	~~~ \\
	\[ H=\begin{pmatrix}
		15 & -2 & 6  \\
		21 & -2 & 9\\
		-28 & 4 & -11
		
	\end{pmatrix}
	\]
	~~~ \\
	$\lambda_{1}=0$ : Valeur propre simple ($\alpha_{1}=1$).\\
	$\lambda_{2}=1$ : Valeur propre double ($\alpha_{2}=2$).\\
	~~~ \\
	$V_{1}=\begin{pmatrix}
		-\frac{1}{2}  \\
		-\frac{3}{4} \\
		1
		
		
	\end{pmatrix}$ ~~~;~~~~~~~~~~~~~~~~~~~~~~~~~~~ \textcolor{red}{$dim E(\lambda_{1})=1=\alpha_{1}$}.\\
	~~~ \\
	~~~ \\
	$V_{2}=\begin{pmatrix}
		1  \\
		7  \\
		0
		
	\end{pmatrix}$ ~~~;~~~$V_{3}=\begin{pmatrix}
		0  \\
		3  \\
		1
		
	\end{pmatrix}$ ~~~;~~~ \textcolor{red}{$dim E(\lambda_{2})=2=\alpha_{2}$}.\\
	
	$H$ est une matrice diagonalisable donc il existe une matrice $D$ diagonale et une matrice $P$ tel que $H=P.D.P^{-1}$ avec:\\
	
	\[ D=\begin{pmatrix}
		0 & 0 & 0  \\
		0 & 1 & 0\\
		0 & 0 & 1
		
	\end{pmatrix} ~~~;~~~P=\begin{pmatrix}
		-\frac{1}{2} & 1 & 0  \\
		-\frac{3}{4} & 7 & 3\\
		1& 0 & 1
		
	\end{pmatrix}
	\]
	\subsection{Application: Calcul de $A^n$}
	$A$ est une matrice carrée diagonalisable, Calculer $A^n$.
	\begin{center}
		$A^2=A.A$ ~~~ or ~~~ $A=P.D.P^{-1}$\\
	\end{center}
	D'où 
	\begin{center}
		$A^{2}=P.D.P^{-1}.P.D.P^{-1}=P.D^2.P^{-1}$\\
		$A^{3}=A^{2}.A=P.D^{2}.P^{-1}.P.D.P^{-1}=P.D^3.P^{-1}$\\
		.\\
		.\\
		.\\
		$A^n=P.D^n.P^{-1}$\\
		$A^{n+1}=A^n.A=P.D^n.P^{-1}.P.D.P^{-1}=P.D^{n+1}.P^{-1}$ 
		
	\end{center}
	
	
	
	
	
\end{document}

